%\documentclass[twocolumn,linenumbers]{aastex631}
\documentclass[iop]{desc-tex/styles/emulateapj}
%\usepackage{desc-tex/styles/lsstdesc_macros}
%\usepackage[utf8]{inputenc}
% \documentclass[iop]{emulateapj}
%\usepackage{calc}
\graphicspath{images/}
%\usepackage{caption}
%\usepackage{subcaption}
\usepackage{amsmath} 
\usepackage{mathtools}
\usepackage[caption=false]{subfig}

\begin{document}
\title{Modeling Lensed Quasars with Neural Posterior Estimation: Complex Mass Models}

\author{
% Primary authors (in order of contribution scale):
{Logan O'Brien},\altaffilmark{1,2}
{Sydney Erickson},\altaffilmark{2,3}
{Phil Marshall},\altaffilmark{2,3}
{Sebastian Wagner-Carena},\altaffilmark{5}
{Simon Birrer},\altaffilmark{4}
{Martin Millon},\altaffilmark{2}
{Padmavathi Venkatraman},\altaffilmark{2}
{Aaron Roodman}\altaffilmark{2,3}
% Secondary authors (alphabetical)
{others (the~LSST~Dark~Energy~Science~Collaboration)}}

\altaffiltext{1}{Department of Physics and Astronomy, San Jose State University, San Jose, CA, USA}
\altaffiltext{2}{Kavli Institute for Particle Astrophysics and Cosmology, Department of Physics, Stanford University, Stanford, CA 94309, USA}
\altaffiltext{3}{SLAC National Accelerator Laboratory, 2575 Sand Hill Road, Menlo Park, CA 94025, USA}
\altaffiltext{4}{Department of Physics, Unversity of Oxford, Keble Road, Oxford, UK}
\altaffiltext{5}{Department of Physics, Stony Brook Unversity, NY, U.S.A.}
\altaffiltext{5}{Flatiron Institute, NY, U.S.A.}
% \altaffiltext{6}{University of California, Davis, CA, U.S.A.}

% The above list contains the primary authors only, so far. Secondary authors (including builders) go after these, alphabetically.

\begin{abstract}

Abstract here.

\end{abstract}

\maketitle

\section{Introduction}


We aim to answer the following questions:
\begin{itemize}

    \item ...

    \item ...

    \item ...


\end{itemize}

To answer these questions, we ...   

In Section \ref{section:background}, we introduce our statistical framework for handling many lenses and explain our modeling assumptions. In Section \ref{section:method}, we describe our NPE modeling tool. Then, in Section \ref{section:data}, we describe the datasets we employ our method on. In Section \ref{section:results} we show the results on our verification test sets. We end with a discussion of results in Section \ref{section:discussion} and conclusions in Section \ref{section:conclusion}.

\section{Background}
\label{section:background}



\section{Method}
\label{section:method}


\section{Data}
\label{section:data}


\section{Results}
\label{section:results}


\section{Discussion}
\label{section:discussion}


\section{Conclusion}
\label{section:conclusion}


We draw the following conclusions:

\begin{itemize}

    \item ...

    \item ...

\end{itemize}



\section{Acknowledgements}

We thank XX for useful discussions.

% Contribution statements:
LO developed new SNPE and HBI code, ran analysis for all training and testing, produced all figures, and wrote the main body of the text. SME helped implement the method, especially the SNPE, and provided feedback on all aspects. PM helped design the statistical framework, helped design the verification tests, and provided feedback on all aspects. 

% Funding acknowldegements:
LO received support from the HEP program...
This work was supported by the U.S. Department of Energy under contract number DE-AC02-76SF00515.
SE acknowledges funding from the NSF GRFP, and Stanford Data Science Scholars.
 

\bibliography{main}

\appendix

\section{Appendix}
\label{appendix}


\end{document}
